
\section{Definitions}
We will start with a few definitions related to matrices.  It helps if you think of them like database tables, so I will use database table imagry in my explanations.
\\
\begin{description}
\item [columns] the fields that define a table; they go left to right, horizontally
\[
\begin{array}{lc}
 \verb|example of columns| & \kbordermatrix{\text{columns}&col_1&col_2&\ldots &col_n\cr
                rows& \vdots & \vdots & \ddots & \vdots\cr
               } \\[15pt]
\end{array}
\]


\item [dimension] numeric description of the number of rows and columns; in math problems, they usually say an "m by n" matrix or "m x n" \\where m = rows and n = columns
\[
\begin{array}{lc}
  \verb|3x3 : m=3, n=3| & \begin{bmatrix}
                    a & b & c \\
                    d & e & f \\
                    g & h & i
                  \end{bmatrix} \\[15pt]
\\
  \verb|2x3 : rows=2, cols=3| & \begin{bmatrix}
                    a & b & c \\
                    d & e & f 
                  \end{bmatrix} \\[15pt]
\\
  \verb|2x4 : m=2, n=4| & \begin{bmatrix}
                    a & b & c & d\\
                    e & f & g & h 
                  \end{bmatrix} \\[15pt]
\end{array}
\]
\item [matrix] an array that has entries in two directions (or more): an "m by n" or "m x n" array


\centerline{\it (see the examples for 'dimension' ) }

\item [multiplication] multiplier x multiplicand = product
\\
\item [reduced row eschelon form (RREF)]  see reference link below
\\
\item [rows] the data that populates the table; each row usually has an entry for each column
\[
\begin{array}{lc}
 \verb|example of rows| & \kbordermatrix{&col_1&col_2&\ldots &col_n\cr
	     row_1&\vdots &  \vdots  & \ldots & \vdots\cr
                row_2& \vdots  &  \vdots & \ldots & \vdots\cr	
                row_m& \vdots & \vdots & \ldots & \vdots\cr
               } \\[15pt]
\end{array}
\]
\item [shape] another term for dimension

\item [vector] an array that has entries in one direction:\\1 row by n columns\\or m rows by 1 column
\[
\begin{array}{lc}
 \verb|row vector| & \kbordermatrix{\text{columns}&col_1&col_2&\ldots &col_n\cr
                1 row& a & b & \ldots & d\cr
               } \\[15pt]
\end{array}
\]
\[
\begin{array}{lc}
 \verb|column vector| & \kbordermatrix{\text{1 column}&col_1\cr
	     row_1&a\cr
                row_2& b\cr	
                row_m& \vdots\cr
               } \\[15pt]
\end{array}
\]


\end{description}
